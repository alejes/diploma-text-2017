\section{Реализация решения}

Для осуществления поддержки динамической типизации для компилятора Kotlin в JVM, необходимо её поддержать с двух сторон:

\begin{enumerate}
    \item Требуется разработка изменений в компиляторе, которые будут в тех местах, где осуществление типизации отложено до момента выполнения, реализовать вызов динаимических инструкций.
    \item Необходимо предоставить библиотеку, которая предоставляет набор загрузочных функций, которые выполняют необходимое динаимическое связывание в требуемых случаях - при вызове методов и при чтении или записи в поля.
\end{enumerate}

\subsection{Реализация поддержки динамических вызовов на стороне компилятора}

Благодаря тому что при компиляции Kotlin в JavaScript, уже поддерживаются динамические вызовы, первая задача решается определение того кода, который мы будем генерировать при компиляции в байт код Java. Мы это будем делать при помощи инструкции \textit{invokedynamic}.
Рассмотрим подробнее её синтаксис.

СИНТАКСИС.

Первым аргументом этой инструкции мы передадим идентификатор действия, согласно таблице \ref{tab:DynamicCallType}. 

\begin{table}[h]
\caption{\label{tab:DynamicCallType}Идентификаторы динамического действия}
\begin{center}
\begin{tabular}{|c|l|}
\hline
Идентификатор	& Требуемое действие \\
\hline
getField & Получение значение поля или свойства класса \\
setField & Установка нового значения поля или свойства класса	\\
invoke &	Вызов функции или объекта умеющего обработать вызов\\
\hline
\end{tabular}
\end{center}
\end{table} 

Вторым аргументом нам необходимо передать ожидаемый тип функции, которую мы хотим получить в результате вызова. Этот тип мы можем составить во время компиляции, путём составления \textit{дескриптора метода}, согласно правилам описанным в главе 4.3 Descriptors \cite{book:yellin1996java}. Несмотря на то, что во время компиляции мы не знаем будущие типы, которые будут лежать в \textit{dynamic}-переменных во время исполнения, мы универсально можно приравнять их к переменным типа \textit{Object}, и использовать его сигнатуру в дескрипторе.