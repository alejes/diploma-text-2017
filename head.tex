% В этом шаблоне используется класс spbau-diploma. Его можно найти и, если требуется, 
% поправить в файле spbau-diploma.cls
\documentclass{spbau-diploma}
\begin{document}


% Год, город, название университета и факультета предопределены,
% но можно и поменять.
% Если англоязычная титульная страница не нужна, то ее можно просто удалить.
\filltitle{ru}{
    chair              = {Кафедра математических и информационных технологий},
    title              = {Поддержка постепенной типизации в языке Kotlin},
    % Здесь указывается тип работы. Возможные значения:
    %   coursework - Курсовая работа
    %   diploma - Диплом специалиста
    %   master - Диплом магистра
    %   bachelor - Диплом бакалавра
    type               = {master},
    position           = {студента},
    group              = 604,
    author             = {Степанов Алексей Макарович},
    supervisorPosition = {к.\,ф.-м.\,н.},
    supervisor         = {Бреслав А.\,А.},
    reviewerPosition   = {ст. преп.},
    reviewer           = {Привалов А.\,И.},
    chairHeadPosition  = {д.\,ф.-м.\,н., профессор},
    chairHead          = {Омельченко А.\,В.},
    % university = {САНКТ-ПЕТЕРБУРГСКИЙ АКАДЕМИЧЕСКИЙ УНИВЕРСИТЕТ},
    % faculty = {Центр высшего образования},
    % city = {Санкт-Петербург},
    % year             = {2013}
}
\filltitle{en}{
    chair              = {Department of Mathematics and Information Technology},
    title              = {Empty subset as closed set},
    author             = {Alexey Stepanov},
    supervisorPosition = {PhD},
    supervisor         = {Andrey Breslav},
    reviewerPosition   = {assistant},
    reviewer           = {Alexander Privalov},
    chairHeadPosition  = {professor},
    chairHead          = {Alexander Omelchenko},
}