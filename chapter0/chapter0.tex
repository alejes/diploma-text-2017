% У введения нет номера главы
\section*{Введение}

Статические и динамические системы типов имеют множество хорошо известных преимуществ и недостатков. В зависимости от задачи, программисту может быть удобно использовать или одну или другую. Использовать преимущества обеих систем типов старается постепенная типизация (gradual typing) \cite{gradual:siek2006gradual}. Путём добавления динамического типа, мы можем обеспечить гибкость в статически типизированном языке. Путём добавления типовой информации появляется возможность добавить некоторые ограничения в динамическом языке.

Одним из примеров статически типизированных языков, является язык Kotlin. Он начал разрабатываться компанией JetBrains в 2011, а в феврале 2016 вышла его первая версия. Kotlin поддерживает компиляцию в JavaScript и байт-код Java.

В рамках данной работы, мы рассмотрим способ внедрения поддержки постепенной типизации в языке Kotlin. Будет предложено поведение языка при работе с динамически типизированным кодом. Мы рассмотрим реализацию данного улучшения, в виде изменений исходного кода компилятора Kotlin. Будет проведено исследование, сравнивающее производительность динамического кода, сгенениророванного представленным решением, со статически типизированным, генерируемым компилятором Kotlin. Также будет произведено сравнение с динамически и статически типизированным кодом, продуцируемым компилятором языка Groovy.
%при котором проверки происходят в момент компиляции  программы, 
%а также с аналогичными решениями в других языках.



%Словарь:
%интерфейс, trait, статический метод