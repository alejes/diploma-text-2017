% У введения нет номера главы
\section*{Введение}

%Статические и динамические системы типов имеют множество хорошо известных преимуществ и недостатков.
В языках программирования широко распространены приёмы статической и динамической типизации. Статическая типизация отличается ранним обнаружением ошибок в программах, широкой поддержкой программными инструментами и возможностью порождать высокопроизводительный код. Динамическая типизация позволяет создавать гибкий код и упрощает написание несложных программ. 
В зависимости от задачи, программисту может быть удобно использовать или один или другой вид типизации. Для объединения преимуществ обоих приёмов, используется постепенная типизация (gradual typing) \cite{gradual:siek2006gradual}. Есть два традиционных способа введения постепенной типизации в язык программирования. Можно обеспечить гибкость в статически типизированном языке добавлением динамического типа, или можно добавить ограничения и проверки в рамках динамически типизированного языка посредством добавления типовой информации.

Язык Kotlin был разработан как статически типизированный.
%Он начал разрабатываться компанией JetBrains в 2011 году, а в феврале 2016 вышла его первая версия. 
Kotlin, в качестве целевой платформы, поддерживает JavaScript и байт-код Java. При компиляции в JavaScript возможно использование динамически типизированного кода, а при компиляции в байт-код Java --- нет.

В рамках данной работы, будет разработана поддержка постепенной типизации в языке Kotlin при компиляции в байт-код Java. 
%Будут рассмотрены детали работы с динамическим кодом в языках Groovy и C\#.  На основании анализа этих деталей, 
В работе будет предложено поведение языка Kotlin при взаимодействии с динамически типизированным кодом. В частности, будет рассмотрен механизм поиска перегрузок методов во время выполнения программы --- процесс выбора из всех методов с требуемым именем, того, который лучше всего подходит для вызова с текущими аргументами. В работе будет описана реализация постепенной компиляции в коде компилятора языка Kotlin. Будет проведено исследование, сравнивающее производительность динамического кода, сгенерированного представленным решением, со статически типизированным, генерируемым компилятором Kotlin. Также будет произведено сравнение производительности генерируемого кода, с динамически и статически типизированным кодом, генерируемым компилятором языка Groovy. 
%Для обеспечения корректности замеров, при проведении сравнений будет использован программный инструмент JMH \cite{java:jmh}.
%при котором проверки происходят в момент компиляции  программы, 
%а также с аналогичными решениями в других языках.



%Словарь:
%интерфейс, trait, статический метод