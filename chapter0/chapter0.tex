% У введения нет номера главы
\section*{Введение}

В настоящий момент Java является одним из самых популярных языков программирования \cite{online:TIOBELanguageIndex}. Программы на Java транслируются в промежуточное представление - байткод Java \cite{book:yellin1996java}. Байткод состоит из набора инструкций, которые интерпертируются с помощью виртуальной машины Java (JVM).

Несмотря на свою популярность, язык Java имеет ряд недостатков, например, требование к обратной совместимости с предыдущими версиями языка, громозкость синтаксиса и медленное развитие. Последнее время, стали появляться новые языки, такие как Scala, Kotlin, Groovy. Они тоже компилируются в Java байткод и используют для выполнения JVM. Некоторые языки также поддерживают и другие платформы в качестве целевой. Новые языки решают часть проблем Java, но иногда создают собственные.

Язык Kotlin начал разрабатываться компанией JetBrains в 2011, а в феврале 2016 вышла его первая версия.

В рамках данной работы, мы рассмотрим улучшение языка Kotlin, которые откладывает некоторые проверки до момента выполнения программы, позволяя программисту более гибко настраивать свою кодовую базу. Будет представленно изменение исходного кода компилятора Kotlin, предназначенное для поддержки данного улучшения. Также будет проведено исследование, сравнивающее производительность представленного решения с текущим, при котором проверки происходят в момент компиляции  программы, а также с аналогичными решениями в других языках.