% У введения нет номера главы
\section*{Введение}

Статические и динамические системы типов имеют множество хорошо известных преимуществ и недостатков. В зависимости от задачи, программисту может быть удобней использовать одну или другую из них. Использовать преимущества обеих систем типов старается постепенная типизация (gradual typing) \cite{gradual:siek2006gradual}. Путём добавления динамического типа, мы можем обеспечить гибкость в статически типизированном языке. Путём добавления типовой информации появляется возможность добавить некоторые ограничения в динамическом языке.



Язык Kotlin начал разрабатываться компанией JetBrains в 2011, а в феврале 2016 вышла его первая версия. Kotlin является статически типизированным и поддерживает компиляцию в JavaScript и Java байт-код.

В рамках данной работы, мы рассмотрим способ поддержки постепенной типизации в языке Kotlin. Будет предложено поведение языка при работе с динамически типизированным кодом. Мы рассмотрим реализацию данного улучшения в виде изменений исходного кода компилятора Kotlin. Будет проведено исследование, сравнивающее производительность представленного решения со статически типизированным, 
%при котором проверки происходят в момент компиляции  программы, 
а также с аналогичными решениями в других языках.



%Словарь:
%интерфейс, trait, статический метод