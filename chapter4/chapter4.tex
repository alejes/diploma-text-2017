\section{Измерения производительности}

Одним из важных критериев использования динамического кода ялвяется его производительность. Коль скоро, множество проверок и вычислений, который делает статически типизированный код во время компиляции, динамическому коду приходится производить во время выполнения, это сильно сказывается на времени его работы.

Важным фактором в любом измерении является степень доверия результату. Код, предназначенный для Java платформы, запускается при помощи виртуальной машины, поэтому нам необходимо учитывать её особенности. Во время выполнения, Java машина может часть байт кода интерпретировать, выполняя инструкцию за инструкцией, а часть  компилировать в машинный код. Машинный код выполняется быстрее чем интерпретируемый, однако требует время на свою генерацию. Во время работы, Java машина накапливает статистику по выполняемому коду, на основании которой она может решать каким способом выполнять каждый участок кода, а также производить различные оптимизации. Часто, чтобы позволить JVM накопить статистическую информацию, перед замером, выполняют несколько, так называемых <<разогревочных запусков>>.

С накоплением достаточного количество статистики, возникает следующая проблема. Часть кода может быть удалено, потмоу что Java машина посчитает что он никак не используется и не влияет на результат работы программы. Также, вызов некоторых методов может быть заменён на результат их вычисления, если во время выполнения станет ясно что результат не зависит от внешнего окружения. Выполнения замеров производительности на таком оптимизированном коде, даёт ложную картину о скорости работы программы.

ПАМЯТЬ

ЗАГРУЗКА ДРУГГИХ ПРОГРАММ

JMH.

\subsection{Окружение при проведении измерений производительности}

\begin{itemize}
    \item \textbf{Операционная система:} Linux version 4.4.0-71-generic (buildd\at lcy01-05) (gcc version 5.4.0 20160609 (Ubuntu 5.4.0-6ubuntu1~16.04.4) ) \#92-Ubuntu SMP Fri Mar 24 12:59:01 UTC 2017.
    \item \textbf{Центральный процессор:}  Intel(R) Core(TM) i7-6700 CPU \at 3.40GHz.
    \item \textbf{Версия JVM:}  Java HotSpot(TM) 64-Bit Server VM (build 25.121-b13, mixed mode).
    \item \textbf{Версия JRE:}  Java(TM) SE Runtime Environment (build 1.8.0\_121-b13).
\end{itemize}